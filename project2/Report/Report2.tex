\documentclass[10pt]{article}

\usepackage{amsfonts}
\usepackage{amsmath}
\usepackage{geometry}
\usepackage{xcolor,graphicx}
%\usepackage{wrapfig}
\usepackage{subcaption}
\usepackage{hyperref}
\usepackage{mathtools}


\providecommand{\e}[1]{\ensuremath{\times 10^{#1}}}

\title{Simulating the Universe}
\author{Najla Alhamadi, Michael Cock, Rebecca Halloran, Brady Metherall, Hector Robinson}

\newgeometry{margin=1in}
\setlength\parindent{0pt}

\begin{document}
\maketitle
\section{Introduction}
	In this project, our goal was to create our own realistic simulation of the universe on a cosmological scale. Our simulation starts from a redshift of 50 (start of the universe), until a redshift of 0 (current time). In addition, our simulation only considers matter and the cosmological constant as the two components of the universe. We do this by using a particle mesh technique to simulate the universe, and solve Poisson's equation to find the accelerations of particles. We step the universe forward by using a leapfrog integrator.
\section{A Very Big Box}
	Our first step was to find the parameters of our current universe, as follows: the current Hubble constant value is 73.8 km/s/Mpc; the density parameter of the cosmological constant is 0.73; the density parameter for matter is 0.27; the total density parameter is 1.02. The critical density of the universe was found to be 9.277\e{-25} kg/m$^3$ and the value of the matter density of the universe is 2.505\e{-25} kg/m$^3$. The current age of the universe was found to be 13923.769 million years. Finally, the age of the universe at z = 50 was found to be 45.813 million years. We took a cubic section of the universe with dimensions of 50x50x50 and calculated the mass of matter inside that cube to be 1.0146\e{46} kg.  
\section{Particle-Mesh Code}
 (write about the writing of the code) \\
	In this step, we built our own Particle-Mesh Code, the input is an initial conditions N-body file and the outputs are snapshots of the system at regular "timestep" intervals. The only parameter that we could manipulate is the time step. After some testing, the value of the time step was chosen to be 0.0025. The code was tested by creating two systems; the first one was to create a perfect grid of particles, evenly spaced. Since there’s no difference in density from the average value, the grid should persist throughout the simulations and that is what is observed in our results in Figure \ref{fig:order}{}. The second system was one with random initial conditions. This should result in the formations of structures in regions with higher initial density and empty space in those with lower initial density. Figure \ref{fig:randomic}{} shows the results of the second system which agrees with the expected results.       
\section{A (somewhat) Realistic Simulation}
(add how the scaling was done)\\ 
	In this step, we ran our code using the initial conditions provided, after scaling the box size the results are presented in Figure 3.   

\section{FoF}
One question to investigate in our simulations is the structures formed, how much of the mass do they contain? It is difficult to define exactly where a structure ends, so to do this we implemented the "friend-of-a-friend" method (FoF). The idea with this method, is that groups are made up of particles within a "linking length" of each other. If two particles have one or more common neighboring particles, they are merged into a group. It checks every particle in the system and links them together into large groups. If the linking length is chosen properly, these groups will correspond to the structures seen in the simulations.\\
To implement this we used a software package called FoF, created by Volker Springel. We used instructions on how to implement the code from LASTRO, which can be found \href{http://obswww.unige.ch/lastro/misc/TP4/doc/rst/Exercices/Ex05.html#extracting-haloes}{here}. This software provided us with infromation the groups' mass, center of mass, and size. We plotted the groups to make sure they matched up \ref{fig:groups}{}. It is seen quite obviously that the groups match up with those in the simulation. The largest two groups had masses of 29% and 22% of the total mass.

\section{Gadget}
GADGET
Gadget  is a very complex system to use. It needs a lot more time to analyze the user guide and the code to really understand what is happening. We were able to download the program and run it with sample parameters that were included in the download. Instead of trying to change our plotting code to take the snapshot files or changing the snapshot files that can be read by our program we were able to find a program that will view the snapshot files. This program is called Gadget filer viewer. It is very simple to use. It opens the file to view them and you can change the colour scheme depending on the particle type. Unfortunately we were unable to play with it long enough to determine how to get a video from it. The following image is form Gadget File Viewer of two galaxies colliding. 

After having the ability to view the snapshots we now want to run the it with our initial conditions that we had using the particle mesh so we can compare our results. We were given initial conditions to use with a parameter file that was changed to be the same as our system. Unfortunately, after many attempts the program was still giving errors about the unit mass in grams that were giving it. We had to abandon hope and attempt to try a different approach that might work better. With the unsuccessful attempt at working with Gadget we attempt to use a program we have used before, dnc. 
	To get dnc to be able to run a collapse of the universe there needs to be some changes. We need to centre a box at (0,0,0) and cut out a sphere with radius 25.  The positions will need to be scaled by the scale factor of the universe, a. This will make up for us not being able to use co-moving coordinates. Initially our system will be smaller than it is today, then it will increase to the size it is today. The initial velocity needs to be different for the initially. Normally we have it set to 0 but it needs to be in a Hubble flow which is the Hubble constant at z=50 times the position. 
\section{answer questions here}


\begin{figure}[htbp]
\centering
\include{Ordered}
\caption{Final state of system with ordered initial conditions. Nothing happened here; all particles remained motionless (Zoom in and you can see the particles!)}
\label{fig:order}
\end{figure}

\begin{figure}[htbp]
\centering
\begin{subfigure}{\textwidth}
\centering
\input{Random2D}
\end{subfigure} \\
\begin{subfigure}{\textwidth}
\centering
\input{Random3D}
\end{subfigure}
\caption{Final state of system with random initial conditions, shown in 2D and 3D. Color corresponds to particle velocity.}
\label{fig:randomic}
\end{figure}

\begin{figure}[htbp]
\centering
\begin{subfigure}{\textwidth}
\centering
\input{Control}
\caption{Top-down view.}
\end{subfigure} \\
\begin{subfigure}{\textwidth}
\centering
\input{Control3D}
\caption{3D view.}
\end{subfigure}
\caption{Control simulation.}
\label{fig:control}
\end{figure}

\begin{figure}[htbp]
\centering
% GNUPLOT: LaTeX picture with Postscript
\begingroup
  \makeatletter
  \providecommand\color[2][]{%
    \GenericError{(gnuplot) \space\space\space\@spaces}{%
      Package color not loaded in conjunction with
      terminal option `colourtext'%
    }{See the gnuplot documentation for explanation.%
    }{Either use 'blacktext' in gnuplot or load the package
      color.sty in LaTeX.}%
    \renewcommand\color[2][]{}%
  }%
  \providecommand\includegraphics[2][]{%
    \GenericError{(gnuplot) \space\space\space\@spaces}{%
      Package graphicx or graphics not loaded%
    }{See the gnuplot documentation for explanation.%
    }{The gnuplot epslatex terminal needs graphicx.sty or graphics.sty.}%
    \renewcommand\includegraphics[2][]{}%
  }%
  \providecommand\rotatebox[2]{#2}%
  \@ifundefined{ifGPcolor}{%
    \newif\ifGPcolor
    \GPcolorfalse
  }{}%
  \@ifundefined{ifGPblacktext}{%
    \newif\ifGPblacktext
    \GPblacktexttrue
  }{}%
  % define a \g@addto@macro without @ in the name:
  \let\gplgaddtomacro\g@addto@macro
  % define empty templates for all commands taking text:
  \gdef\gplbacktext{}%
  \gdef\gplfronttext{}%
  \makeatother
  \ifGPblacktext
    % no textcolor at all
    \def\colorrgb#1{}%
    \def\colorgray#1{}%
  \else
    % gray or color?
    \ifGPcolor
      \def\colorrgb#1{\color[rgb]{#1}}%
      \def\colorgray#1{\color[gray]{#1}}%
      \expandafter\def\csname LTw\endcsname{\color{white}}%
      \expandafter\def\csname LTb\endcsname{\color{black}}%
      \expandafter\def\csname LTa\endcsname{\color{black}}%
      \expandafter\def\csname LT0\endcsname{\color[rgb]{1,0,0}}%
      \expandafter\def\csname LT1\endcsname{\color[rgb]{0,1,0}}%
      \expandafter\def\csname LT2\endcsname{\color[rgb]{0,0,1}}%
      \expandafter\def\csname LT3\endcsname{\color[rgb]{1,0,1}}%
      \expandafter\def\csname LT4\endcsname{\color[rgb]{0,1,1}}%
      \expandafter\def\csname LT5\endcsname{\color[rgb]{1,1,0}}%
      \expandafter\def\csname LT6\endcsname{\color[rgb]{0,0,0}}%
      \expandafter\def\csname LT7\endcsname{\color[rgb]{1,0.3,0}}%
      \expandafter\def\csname LT8\endcsname{\color[rgb]{0.5,0.5,0.5}}%
    \else
      % gray
      \def\colorrgb#1{\color{black}}%
      \def\colorgray#1{\color[gray]{#1}}%
      \expandafter\def\csname LTw\endcsname{\color{white}}%
      \expandafter\def\csname LTb\endcsname{\color{black}}%
      \expandafter\def\csname LTa\endcsname{\color{black}}%
      \expandafter\def\csname LT0\endcsname{\color{black}}%
      \expandafter\def\csname LT1\endcsname{\color{black}}%
      \expandafter\def\csname LT2\endcsname{\color{black}}%
      \expandafter\def\csname LT3\endcsname{\color{black}}%
      \expandafter\def\csname LT4\endcsname{\color{black}}%
      \expandafter\def\csname LT5\endcsname{\color{black}}%
      \expandafter\def\csname LT6\endcsname{\color{black}}%
      \expandafter\def\csname LT7\endcsname{\color{black}}%
      \expandafter\def\csname LT8\endcsname{\color{black}}%
    \fi
  \fi
    \setlength{\unitlength}{0.0500bp}%
    \ifx\gptboxheight\undefined%
      \newlength{\gptboxheight}%
      \newlength{\gptboxwidth}%
      \newsavebox{\gptboxtext}%
    \fi%
    \setlength{\fboxrule}{0.5pt}%
    \setlength{\fboxsep}{1pt}%
\begin{picture}(7674.00,5760.00)%
    \gplgaddtomacro\gplbacktext{%
      \csname LTb\endcsname%
      \put(814,704){\makebox(0,0)[r]{\strut{}$10$}}%
      \put(814,1236){\makebox(0,0)[r]{\strut{}$20$}}%
      \put(814,1769){\makebox(0,0)[r]{\strut{}$30$}}%
      \put(814,2301){\makebox(0,0)[r]{\strut{}$40$}}%
      \put(814,2833){\makebox(0,0)[r]{\strut{}$50$}}%
      \put(814,3366){\makebox(0,0)[r]{\strut{}$60$}}%
      \put(814,3898){\makebox(0,0)[r]{\strut{}$70$}}%
      \put(814,4430){\makebox(0,0)[r]{\strut{}$80$}}%
      \put(814,4963){\makebox(0,0)[r]{\strut{}$90$}}%
      \put(814,5495){\makebox(0,0)[r]{\strut{}$100$}}%
      \put(946,484){\makebox(0,0){\strut{}$0$}}%
      \put(2212,484){\makebox(0,0){\strut{}$0.2$}}%
      \put(3478,484){\makebox(0,0){\strut{}$0.4$}}%
      \put(4745,484){\makebox(0,0){\strut{}$0.6$}}%
      \put(6011,484){\makebox(0,0){\strut{}$0.8$}}%
      \put(7277,484){\makebox(0,0){\strut{}$1$}}%
    }%
    \gplgaddtomacro\gplfronttext{%
      \csname LTb\endcsname%
      \put(176,3099){\rotatebox{-270}{\makebox(0,0){\strut{}Average Acceleration}}}%
      \put(4111,154){\makebox(0,0){\strut{}$a$}}%
    }%
    \gplbacktext
    \put(0,0){\includegraphics{Accel}}%
    \gplfronttext
  \end{picture}%
\endgroup

\label{fig:accel}
\caption{Average acceleration of particles throughout the simulation.}
\end{figure}

\begin{figure}[htbp]
\centering
\include{1Ncells}
\caption{$N_c = n^{1/3}$}
\label{fig:1ncells}
\end{figure}

\begin{figure}[htbp]
\centering
\include{3mass}
\caption{$\Omega_m = 3.0$, $\Omega_k = \Omega_\Lambda = 0$}
\label{fig:3mass}
\end{figure}

\begin{figure}[htbp]
\centering
\include{N64}
\caption{$n = 64^3$}
\label{fig:n64}
\end{figure}

\begin{figure}[htbp]
\centering
\include{1boxsize}
\caption{boxsize = 1}
\label{fig:1boxsize}
\end{figure}

\begin{figure}[htbp]
\centering
\includegraphics{Control_Groups.pdf}
\caption{Groups}
\label{fig:groups}
\end{figure}

\end{document}
