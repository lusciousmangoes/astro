\documentclass[10pt]{article}

\usepackage{amsfonts}
\usepackage{amsmath}
\usepackage{geometry}
\usepackage{xcolor,graphicx}
%\usepackage{wrapfig}
\usepackage{subcaption}
\usepackage{hyperref}
\usepackage{mathtools}


\providecommand{\e}[1]{\ensuremath{\times 10^{#1}}}

\title{Simulating the Universe}
\author{Najla Alhamadi, Michael Cock, Rebecca Halloran, Brady Metherall, Hector Robinson}

\newgeometry{margin=1in}
\setlength\parindent{0pt}

\begin{document}
\maketitle
\section{Introduction}
	In this project, our goal was to create our own realistic simulation of the universe on a cosmological scale. Our simulation starts from a redshift of 50 (beginnings of the universe), until a redshift of 0 (current time). In addition, our simulation only considers matter and the cosmological constant as the two components of the universe. We do this by using a particle mesh technique to simulate the universe, and solve Poisson's equation to find the accelerations of particles. We step the universe forward by using a leapfrog integrator.
	
\section{A Very Big Box}
	Our first step was to find the parameters of our current universe, as follows: the current Hubble constant value is 73.8 km/s/Mpc; the density parameter of the cosmological constant is 0.73; the density parameter for matter is 0.27; the total density parameter is 1.02. The critical density of the universe was found to be 9.277\e{-25} kg/m$^3$ and the value of the matter density of the universe is 2.505\e{-25} kg/m$^3$. The current age of the universe was found to be 13'923.769 million years. Finally, the age of the universe at z = 50 was found to be 45.813 million years. We took a cubic section of the universe with dimensions of 50 x 50 x 50 Mpc$^3$ and calculated the mass of matter inside that cube to be 1.0146\e{46} kg.  
	
\section{Particle-Mesh Code}

	In this step, we built our own Particle-Mesh Code, the input is an initial conditions N-body file and the outputs are snapshots of the system at regular ``timestep" intervals. The only parameter that we could manipulate is the time step. After some testing, the value of the time step was chosen to be 0.0025. We create initial conditions for particles in the system. Dividing the system into boxes can give us a particle density per box. Using this density, we can find the potential. The code we wrote goes through each particle, determine which box it is in and increments that index by one until we have gone through all the particles. Doing so gives us the density of each cell. We can get the perturbation of the density per cell from the normal. Taking the Fast Fourier Transform of the perturbation we can get the Fast Fourier Transform of the potential. To get the actual potential we must do the inverse Fast Fourier Transform. From the potential, we are able to get the acceleration associated with each cell, then update the positions of the particles in those cells based on the accelerations.\\ 
	The code was tested by creating two systems; the first one was to create a perfect grid of particles, evenly spaced. Since there’s no difference in density from the average value, the grid should persist throughout the simulations and that is what is observed in our results in Figure \ref{fig:order}{}. The second system was one with random initial conditions. This should result in the formations of structures in regions with higher initial density and empty space in those with lower initial density. Figure \ref{fig:randomic}{} shows the results of the second system which agrees with the expected results.\\
	We ran into several problems while writing this section of the code. One method of troubleshooting we used was to look at the average acceleration of the particles. Based off of this, we would (hopefully) tell what parts of our code were producing erroneous results, and make the appropriate changes. We plotted the average accelerations of the particles for our simulation, which can be seen in Figure \ref{fig:accel}{}. The weird behaviour in the curve corresponds to the formation of complicated structures in the system.




\section{A (somewhat) Realistic Simulation}
	In this step, we were given a set of initial conditions. The positions in this set of conditions did not match our universe size, so we had to scale them up to match our universe size. We did this by multiplying by the ratios of box sizes. After doing this, we ran our code based off of the altered initial conditions; the results can be seen in Figure \ref{fig:control}{}. It can clearly be seen that there are regions where large structures have been formed, and that there are vast regions where there are very few particles remaining.
	This simulation was more organized than our initial conditions, but less organized than the perfectly ordered system.
	
	
	
\section{FoF}
	One of the questions we should ask about our simulation is about the structures formed; how much mass do they contain? It is difficult to define exactly where a structure ends, so to do this we implemented the ``friend-of-a-friend" method (FoF). The idea with this method, is that groups are made up of particles within a``linking length" of each other. If two particles have one or more common neighbouring particles within the linking length, they are merged into a group. It checks every particle in the system and links them together into large groups. If the linking length is chosen properly, these groups will correspond to the structures seen in the simulations.\\
	To implement this, we used a software package called FoF, created by Volker Springel. We used instructions on how to implement the code from LASTRO (University of Geneva), which can be found \href{http://obswww.unige.ch/lastro/misc/TP4/doc/rst/Exercices/Ex05.html#extracting-haloes}{here}. The software used a different file format than our code, so we needed to convert to the Gadget format. This software provided us with the groups' mass, center of mass, and size. We plotted the groups as seen in Figure \ref{fig:groups}{} to make sure they matched up with those in our previous simulation. It is seen quite obviously that the groups match up with those in Figure \ref{fig:control}. The largest two groups had masses of 29\% and 22\% of the total mass.


\section{Other Questions}
	Another question we should ask is how the size of the universe affects our simulation. As can be seen in Figure \ref{fig:1boxsize}{}, the size of the universe has been shrunk to 1Mpc$^3$. We had to rescale our initial conditions to fit into this size universe, but all other conditions remained the same. The effect of this, was that all the particles collapsed into one super massive structure. The reason for this, is that there is not enough space in between particles initially for any empty regions to form. All of the separate structure formation seen in the previous simulation can not seen here.\\

	A further question we can ask is what would occur if we were to change $\Omega_{m,0}$, or other parameters? Figure \ref{fig:3mass} shows our system when the density parameter for matter was set equal to 3.0 and the other density parameters were set equal to zero. Increasing the density parameter of matter will result in the increase of particles’ mass and setting the cosmological constant density parameter equal to zero means that our system only contains matter. Comparing our plot with the plot obtained using the initial conditions; we can see that both conditions had similar results. Although we would expect our two conditions to have different results, our two systems are almost identical. Due to the limited time constraint, further examination of the effect of changing the parameter density of our universe’s contents was not probable.\\

	The next question asked is what would occur if we varied the size of our grid. Varying the number of cells in our box will affect the distribution of particles in our system and therefore will affect the long-term structure of our system. In Figure \ref{fig:1ncells}, we can observe that the particles are more spread out since there are more cells to fill. In addition, when we compare the plots we can see that in our initial condition plot we have some large structure which can corresponds to galaxies and galaxies clusters whereas this is not observed in the plot with the larger number of cells.\\
	
	A final alteration we can make is to increase the number of particles in our simulations. Doing this will increase the probability of interactions between the particles, since the box is still the same size. Figure \ref{fig:n64} shows our system with the increased number of particles. As we can observe the particles almost formed a one large structure, which can be explained by the fact that the average distance between the particles have decreased by the increase in the number of particles. The decrease in distance have caused the particles to form one large structure instead of forming multiple smaller structures.   



\section{GADGET}
	One further item we decided to look at was to see how our model compared to an industry used system. To do so, we looked into using the GADGET (GAlaxies with Dark matter and Gas intEracT) system.
	GADGET is a very complex system to use. We needs a lot more time to analyze the user guide and the code to really understand what is happening. We were able to download the program and run it with sample parameters that were included in the download. Instead of trying to change our plotting code to take the snapshot files (or vice versa), we were able to find a program that will view the snapshot files. This program ``GADGET FILE VIEWER" is very simple to use. It opens the file in a GUI to view them and the image can be manipulated by changing the colour scheme depending on the particle type. The following image is form Gadget File Viewer of two galaxies colliding.

\begin{figure*}[h]
\includegraphics[width=4.5in]{gadget.png}
\centering
\end{figure*}


	Now that we have the ability to view the snapshots, we now want to run GADGET with our initial conditions that we used for our particle mesh code. This way we can compare our results to the results of GADGET. We were given initial conditions to use with a parameter file that was changed to be the same as our system. Unfortunately, after many attempts the program was still giving errors about the unit mass in grams that we were giving it. We had to abandon hope and attempt to try a different approach that might work better. With the unsuccessful attempt at working with Gadget we attempt to use a program we have used before, DNC.\\
	To get DNC to be able to run a collapse of the universe there needs to be some changes. We need to centre a box at (0,0,0) and cut out a sphere with radius 25.  The positions will need to be scaled by the scale factor of the universe, a. This will make up for us not being able to use co-moving coordinates. Initially our system will be smaller than it is today, then it will increase to the size it is today. The initial velocity needs to be different to incorporate the expansion of the universe. Normally we have it set to 0 but it needs to be in a Hubble flow which is the Hubble constant at z=50 times the position.
	After doing all of this, the system did not perform as expected. The particles in the universe simply collapsed, as would be expected if there was no expansion whatsoever. Yet again, unfortunate time constraints did not allow for us to correct this problem.


\section{Discussion}
	In this project we successfully simulated the large-scale structure of the Universe. We did this with a particle-mesh code. We implemented some FoF code to identify the structures formed in our simulations. There were several difficulties that we ran into along the way, the main issue being time constraints. Had we had more time, we may have been able to iron out some of the other issues that remain. 



\begin{figure}[htbp]
\centering
\include{Ordered}
\caption{Final state of system with ordered initial conditions. Nothing happened here; all particles remained motionless (Zoom in and you can see the particles!)}
\label{fig:order}
\end{figure}

\begin{figure}[htbp]
\centering
\begin{subfigure}{\textwidth}
\centering
\input{Random2D}
\end{subfigure} \\
\begin{subfigure}{\textwidth}
\centering
\input{Random3D}
\end{subfigure}
\caption{Final state of system with random initial conditions, shown in 2D and 3D. Color corresponds to particle velocity.}
\label{fig:randomic}
\end{figure}

\begin{figure}[htbp]
\centering
% GNUPLOT: LaTeX picture with Postscript
\begingroup
  \makeatletter
  \providecommand\color[2][]{%
    \GenericError{(gnuplot) \space\space\space\@spaces}{%
      Package color not loaded in conjunction with
      terminal option `colourtext'%
    }{See the gnuplot documentation for explanation.%
    }{Either use 'blacktext' in gnuplot or load the package
      color.sty in LaTeX.}%
    \renewcommand\color[2][]{}%
  }%
  \providecommand\includegraphics[2][]{%
    \GenericError{(gnuplot) \space\space\space\@spaces}{%
      Package graphicx or graphics not loaded%
    }{See the gnuplot documentation for explanation.%
    }{The gnuplot epslatex terminal needs graphicx.sty or graphics.sty.}%
    \renewcommand\includegraphics[2][]{}%
  }%
  \providecommand\rotatebox[2]{#2}%
  \@ifundefined{ifGPcolor}{%
    \newif\ifGPcolor
    \GPcolorfalse
  }{}%
  \@ifundefined{ifGPblacktext}{%
    \newif\ifGPblacktext
    \GPblacktexttrue
  }{}%
  % define a \g@addto@macro without @ in the name:
  \let\gplgaddtomacro\g@addto@macro
  % define empty templates for all commands taking text:
  \gdef\gplbacktext{}%
  \gdef\gplfronttext{}%
  \makeatother
  \ifGPblacktext
    % no textcolor at all
    \def\colorrgb#1{}%
    \def\colorgray#1{}%
  \else
    % gray or color?
    \ifGPcolor
      \def\colorrgb#1{\color[rgb]{#1}}%
      \def\colorgray#1{\color[gray]{#1}}%
      \expandafter\def\csname LTw\endcsname{\color{white}}%
      \expandafter\def\csname LTb\endcsname{\color{black}}%
      \expandafter\def\csname LTa\endcsname{\color{black}}%
      \expandafter\def\csname LT0\endcsname{\color[rgb]{1,0,0}}%
      \expandafter\def\csname LT1\endcsname{\color[rgb]{0,1,0}}%
      \expandafter\def\csname LT2\endcsname{\color[rgb]{0,0,1}}%
      \expandafter\def\csname LT3\endcsname{\color[rgb]{1,0,1}}%
      \expandafter\def\csname LT4\endcsname{\color[rgb]{0,1,1}}%
      \expandafter\def\csname LT5\endcsname{\color[rgb]{1,1,0}}%
      \expandafter\def\csname LT6\endcsname{\color[rgb]{0,0,0}}%
      \expandafter\def\csname LT7\endcsname{\color[rgb]{1,0.3,0}}%
      \expandafter\def\csname LT8\endcsname{\color[rgb]{0.5,0.5,0.5}}%
    \else
      % gray
      \def\colorrgb#1{\color{black}}%
      \def\colorgray#1{\color[gray]{#1}}%
      \expandafter\def\csname LTw\endcsname{\color{white}}%
      \expandafter\def\csname LTb\endcsname{\color{black}}%
      \expandafter\def\csname LTa\endcsname{\color{black}}%
      \expandafter\def\csname LT0\endcsname{\color{black}}%
      \expandafter\def\csname LT1\endcsname{\color{black}}%
      \expandafter\def\csname LT2\endcsname{\color{black}}%
      \expandafter\def\csname LT3\endcsname{\color{black}}%
      \expandafter\def\csname LT4\endcsname{\color{black}}%
      \expandafter\def\csname LT5\endcsname{\color{black}}%
      \expandafter\def\csname LT6\endcsname{\color{black}}%
      \expandafter\def\csname LT7\endcsname{\color{black}}%
      \expandafter\def\csname LT8\endcsname{\color{black}}%
    \fi
  \fi
    \setlength{\unitlength}{0.0500bp}%
    \ifx\gptboxheight\undefined%
      \newlength{\gptboxheight}%
      \newlength{\gptboxwidth}%
      \newsavebox{\gptboxtext}%
    \fi%
    \setlength{\fboxrule}{0.5pt}%
    \setlength{\fboxsep}{1pt}%
\begin{picture}(7674.00,5760.00)%
    \gplgaddtomacro\gplbacktext{%
      \csname LTb\endcsname%
      \put(814,704){\makebox(0,0)[r]{\strut{}$10$}}%
      \put(814,1236){\makebox(0,0)[r]{\strut{}$20$}}%
      \put(814,1769){\makebox(0,0)[r]{\strut{}$30$}}%
      \put(814,2301){\makebox(0,0)[r]{\strut{}$40$}}%
      \put(814,2833){\makebox(0,0)[r]{\strut{}$50$}}%
      \put(814,3366){\makebox(0,0)[r]{\strut{}$60$}}%
      \put(814,3898){\makebox(0,0)[r]{\strut{}$70$}}%
      \put(814,4430){\makebox(0,0)[r]{\strut{}$80$}}%
      \put(814,4963){\makebox(0,0)[r]{\strut{}$90$}}%
      \put(814,5495){\makebox(0,0)[r]{\strut{}$100$}}%
      \put(946,484){\makebox(0,0){\strut{}$0$}}%
      \put(2212,484){\makebox(0,0){\strut{}$0.2$}}%
      \put(3478,484){\makebox(0,0){\strut{}$0.4$}}%
      \put(4745,484){\makebox(0,0){\strut{}$0.6$}}%
      \put(6011,484){\makebox(0,0){\strut{}$0.8$}}%
      \put(7277,484){\makebox(0,0){\strut{}$1$}}%
    }%
    \gplgaddtomacro\gplfronttext{%
      \csname LTb\endcsname%
      \put(176,3099){\rotatebox{-270}{\makebox(0,0){\strut{}Average Acceleration}}}%
      \put(4111,154){\makebox(0,0){\strut{}$a$}}%
    }%
    \gplbacktext
    \put(0,0){\includegraphics{Accel}}%
    \gplfronttext
  \end{picture}%
\endgroup

\caption{Average acceleration of particles throughout the simulation.}
\label{fig:accel}
\end{figure}


\begin{figure}[htbp]
\centering
\begin{subfigure}{\textwidth}
\centering
\input{Control}
\caption{Top-down view.}
\end{subfigure} \\
\begin{subfigure}{\textwidth}
\centering
\input{Control3D}
\caption{3D view.}
\end{subfigure}
\caption{Control simulation (initial conditions provided to us).}
\label{fig:control}
\end{figure}


\begin{figure}[htbp]
\centering
\includegraphics[width=6.2in]{Control_Groups.pdf}
\caption{Groups}
\label{fig:groups}
\end{figure}


\begin{figure}[htbp]
\centering
\include{1boxsize}
\caption{boxsize = 1}
\label{fig:1boxsize}
\end{figure}

\begin{figure}[htbp]
\centering
\include{3mass}
\caption{$\Omega_m = 3.0$, $\Omega_k = \Omega_\Lambda = 0$}
\label{fig:3mass}
\end{figure}

\begin{figure}[htbp]
\centering
\include{1Ncells}
\caption{$N_c = n^{1/3}$}
\label{fig:1ncells}
\end{figure}

\begin{figure}[htbp]
\centering
\include{N64}
\caption{$n = 64^3$}
\label{fig:n64}
\end{figure}




\end{document}
